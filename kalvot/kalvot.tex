\documentclass[finnish]{beamer}

\usepackage{babel}
\usepackage[utf8]{inputenc}
\usepackage[T1]{fontenc}
\usepackage{amsmath}
\usepackage{amssymb}
\usepackage{wasysym}
\usepackage[official]{eurosym}

\newcommand{\labra}{Ohjelmoinnin harjoitustyö}
\institute{Helsingin Yliopisto, TKTL}
\author{Atte Lassila, Kristian Wahlroos}
\date{14. maaliskuuta - 06. toukokuuta}

% Valitaan teema ja poistetaan navigointisymbolit
\usetheme{Singapore}
\setbeamertemplate{navigation symbols}{}

\begin{document}
	\frame{\titlepage}
	
	\section{Yleistä kurssista}
	
	\begin{frame}
		\frametitle{Ohjaajat}
		
		Kurssin ohjaajat:
		\begin{itemize}
			\item Atte Lassila
			\begin{itemize}
				\item atte.lassila@cs.helsinki.fi
				\item serdion @ IRCNet
			\end{itemize}
			\item Kristian Wahlroos
			\begin{itemize}
				\item kristian.wahlroos@cs.helsinki.fi
				\item wakr @ IRCNet
			\end{itemize}
		\end{itemize}
		
		Ota yhteyttä sähköpostilla tai irkissä (highlight!). \\
		Irkkikanava: \#javalabra @ IRCnet \\
		Kurssisivu: \texttt{https://github.com/javaLabra/Javalabra2016-4}
	\end{frame}
	
	\begin{frame}
		\frametitle{Mistä kyse?}

		Kurssilla \textbf{ohjelmoidaan}, \textbf{testataan}, \textbf{dokumentoidaan}... \\
		Jokainen näistä tärkeä -- jonkin laiminlyönti johtaa hylkäämiseen \\
		Aihe on omavalintainen, mutta älä ota liian suurta aihetta!
	\end{frame}
	
	\begin{frame}
		\frametitle{Esitietovaatimukset}
		
		Kurssilla on pakollisia esitietovaatimuksia:
		\begin{itemize}
			\item Ohjelmoinnin jatkokurssi
			\item Ohjelmistotekniikan menetelmät
		\end{itemize}
		
		Tule juttelemaan luennon jälkeen, jos: 
		\begin{itemize}
			\item Olet käynyt kurssit, mutta olet esim. ollut poissa tentin aikana
			\item Mikä tahansa muu puute esitietovaatimuksiin liittyen
		\end{itemize}
	\end{frame}
	
	\section{Kurssin sisältö ja suoritus}
	
	\begin{frame}
		\frametitle{Kurssin sisältö}

		Vaatimuksia:
		\begin{itemize}
			\item Toteutuskielenä \textbf{Java}
			\item Kolmannen osapuolen kirjastoja saa käyttää (muista lisenssi)
			\item Käytössä Git-versionhallinta sekä GitHub
			\item Harjoitustyö on yksilötyö!
			\item Ohjelman on toimittava laitoksen tietokoneilla
			\item Tuloksena suoritettava ohjelma
			\item Ei pelkkä kirjasto tai muu suorittamattomissa oleva läjä koodia
			\item Ohjelmalla oltava käyttöliittymä, graafinen, esim Swing
		\end{itemize}
	\end{frame}

	\begin{frame}
		\frametitle{Testaus, Maven, Checkstyle ja PIT}
		
		
		
		Lisäksi edellytetään:
		\begin{itemize}
			\item Automaattisen projektinhallintatyökalun käyttöä (Maven)
			\item Kattavaa automaattista testausta (JUnit)
			\begin{itemize}
				\item eli rivikattavuus, mutaatiotestaus (PIT)
			\end{itemize}
			\item Laadukasta koodia, Clean code (Checkstyle)
		\end{itemize}

		
	\end{frame}
	
	\begin{frame}
		\frametitle{Arvosana}
		
		Arvosteluperusteet (max 60p):
		\begin{itemize}
			\item Aikataulun noudattaminen (12 pistettä)
			\item Dokumentaatio (10 pistettä)
			\item Testaus (10 pistettä)
			\item Toteutus (25 pistettä)
			\item Katselmoinnit (3 pistettä)
		\end{itemize}
		
		\begin{tabular}{|c|c|c|c|c|c|}
			\hline Arvosana & 1 & 2 & 3 & 4 & 5 \\ 
			\hline Pisteet & 30 & 36 & 42 & 48 & 54 \\ 
			\hline 
		\end{tabular} 
	\end{frame}
	
	\begin{frame}
		\frametitle{Aiheen valinta}
		
		Esimerkkiaiheita:
		\begin{itemize}
			\item Graafinen laskin
			\item Tekstieditori ("Notepad-klooni")
			\item Keräilykorttien hallintajärjestelmä
			\item Sudoku-ratkaisin
			\item Piirto-ohjelma ("Paint-klooni")
			\item Pong, Asteroids, Space Invaders, Pac-Man  
			\item Shakki, Miinaharava, Muistipeli, Ristinolla
		\end{itemize}
		
		Tällä kurssilla ei ole tärkeää:
		\begin{itemize}
			\item Tekoäly, Grafiikka, Tietoturva, Tehokkuus
			\item \textbf{mutta toimivuus, dokumentointi ja kattavuus on!}
		\end{itemize}
	\end{frame}
	
	\begin{frame}
		\frametitle{Palautuksien takarajat}

		Kurssilla tiukka deadline:
		\begin{itemize}
			\item Deadlinestä annetaan 0-2 pistettä kunkin deadlinen tehtävien mukaisesti, laiminlyönti lasketaan keskeytykseksi
			\item Ensimmäinen deadline \textbf{jo tänä perjantaina}!
			\item Palautukset tehdään pushaamalla projektin kunkin hetkinen tilanne GitHubiin (Ei sähköpostipalautuksia)
			\item Suuri osa pisteistä -- ja siten arvosanasta -- tulevat deadlinejen perusteella (12/60p)
			\item Ohjaajat antavat palautetta edistymisestä joka deadlinen jälkeen -- perusteellisempaa palautetta kannattaa tulla pyytämään pajasta.
		\end{itemize}
	\end{frame}
	
	\begin{frame}
		\frametitle{Paja-ohjaus}

		Kurssin aikana on ohjausta, eli pajaa:
		\begin{itemize}
			\item Täysin vapaaehtoista, mutta hyödyllistä
			\item Exactumissa luokassa \textbf{BK107} (ns. alapaja)
			\item Paras väylä saada apua ja palautetta ohjaajilta
			\item IRC ei ole virallinen tiedonlähde, vaikka onkin kätevä
		\end{itemize}
	\end{frame}
	
	\begin{frame}
		\frametitle{Koodikatselmoinnit}
		
		Kaksi vapaaehtoista koodikatselmointia:
		\begin{itemize}
			\item Jokainen opiskelija saa toisen opiskelijan projektin katselmoitavaksi
			\item Opiskelijat kirjoittavat palautetta toisen projektista
			\item Tarkoitus oppia lukemaan ja ymmärtämään toisten koodia
			\item Katselmoinneista saa 0-1.5 pistettä/katselmointi.
		\end{itemize}
		Katselmoinnit ovat 3. ja 5. deadlinen yhteydessä.
	\end{frame}
	
	\begin{frame}
		\frametitle{Demotilaisuus}
		\begin{itemize}
			\item Kurssin lopuksi \textbf{pakollinen demotilaisuus}
			\begin{itemize}
				\item Jokainen opiskelija esittelee muille projektiaan 3-5 minuuttia
				\item Opiskelijat paikalla koko demotilaisuuden ajan
				\item Harjoitustyön ei tarvitse olla demossa vielä aivan valmis
				\item Muodollinen päätös kurssille
			\end{itemize}
			\item Kurssilla \textbf{ei ole kurssikoetta}
		\end{itemize}
	\end{frame}
	
	\section{Yhteenveto}
	
	\begin{frame}
		\frametitle{Motivaatio}

		Yksi parhaista kursseista
		\begin{itemize}
			\item Voit toteuttaa mitä itse haluat!
			\item Olet itse vastuussa projektisi etenemisestä
		\end{itemize}		
		
		Jos jäät jumiin tule juttelemaan ohjaajille
		\begin{itemize}
			\item Paja järjestetään teitä varten!
		\end{itemize}
		
		Kurssin keskeytys?
		\begin{itemize}
			\item Tavallisesti kurssien keskeyttämisestä ei juuri seurauksia -- harjoitustyöt ovat poikkeus
			\item Kurssille pääsy vaikeutuu keskeyttämisen jälkeen
		\end{itemize}
	\end{frame}
	
	\begin{frame}
		\frametitle{Kiitokset}

		{\LARGE \textbf{Tervetuloa kurssille!}}
		\begin{itemize}
			\item Kaikki tarvittava löytyy kurssisivulta
			\begin{itemize}
				\item \texttt{https://github.com/javaLabra/Javalabra2016-4}
			\end{itemize}
			\item Rekisteröikää Labtool ja seuratkaa palautetta projektistanne
			\begin{itemize}
				\item \texttt{http://tktl-labtool.herokuapp.com/register}
			\end{itemize}
			\item Ohjaajat jäävät paikalle luennon jälkeen
			\begin{itemize}
				\item Tervetuloa kyselemään tai pyytämään apua 
			\end{itemize}
		\end{itemize}
		\textbf{Ensimmäinen palautus Perjantaina 18.03.2016, klo 23:59}
		
	\end{frame}
	
\end{document}